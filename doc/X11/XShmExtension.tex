\documentclass{article}
\title{
	  MIT-SHM--The MIT Shared Memory Extension \\
	   How the shared memory extension works \\
}
\author{Jonathan Corbet\\
	      Atmospheric Technology Division\\
	  National Center for Atmospheric Research\\
		    corbet@ncar.ucar.edu\\
	   Formatted and edited for release 5 by\\
		       Keith Packard\\
		      MIT X Consortium\\
}
\begin{document}
\maketitle
\begin{abstract}
	   This document briefly describes how to use
     the MIT-SHM shared memory extension.  I have tried
     to make it accurate, but it would not surprise me
     if some errors remained.  If you find anything
     wrong, do let me know and I will incorporate the
     corrections.  Meanwhile, please take this document
     ``as is''--an improvement over what was there
     before, but certainly not the definitive word.
\end{abstract}



\section{REQUIREMENTS}
The shared memory extension is provided only by some X
servers.  To find out if your server supports the extension,
use xdpyinfo(1).  In particular, to be able to use this
extension, your system must provide the SYSV shared memory
primitives.  There is not an mmap-based version of this
extension.  To use shared memory on Sun systems, you must
have built your kernel with SYSV shared memory enabled --
which is not the default configuration.	 Additionally, the
shared memeory maximum size will need to be increased on
both Sun and Digital systems; the defaults are far too small
for any useful work.

\section{WHAT IS PROVIDED}
The basic capability provided is that of shared memory XIm-
ages.  This is essentially a version of the ximage interface
where the actual image data is stored in a shared memory
segment, and thus need not be moved through the Xlib inter-
process communication channel.	For large images, use of
this facility can result in some real performance increases.
Additionally, some implementations provided shared memory
pixmaps.  These are 2 dimensional arrays of pixels in a for-
mat specified by the X server, where the image data is
stored in the shared memory segment.  Through use of shared
memory pixmaps, it is possible to change the contents of
these pixmaps without using any Xlib routines at all.
Shared memory pixmaps can only be supported when the X
server can use regular virtual memory for pixmap data; if
the pixmaps are stored in some magic graphics hardware, your
application will not be able to share them with the server.
Xdpyinfo(1) doesn't print this particular nugget of informa-
tion.

\section{HOW TO USE THE SHARED MEMORY EXTENSION}
Code which uses the shared memory extension must include a
number of header files:
\begin{verbatim}
     # include <X11/Xlib.h>	   /* of course */
     # include <sys/ipc.h>
     # include <sys/shm.h>
     # include <X11/extensions/XShm.h>
\end{verbatim}

Of course, if the system you are building on does not sup-
port shared memory, the file XShm.h may not be present.	 You
may want to make liberal use of \#ifdefs.
Any code which uses the shared memory extension should first
check to see that the server provides the extension.  You
could always be running over the net, or in some other envi-
ronment where the extension will not work.  To perform this
check, call either
\begin{verbatim}
     Status XShmQueryExtension (display)
	     Display *display
\end{verbatim}

or

\begin{verbatim}
     Status XShmQueryVersion (display, major, minor, pixmaps)
	     Display *display;
	     int *major, *minor;
	     Bool *pixmaps
\end{verbatim}

Where ``display'' is, of course, the display on which you
are running.  If the shared memory extension may be used,
the return value from either function will be True; other-
wise your program should operate using conventional Xlib
calls.	When the extension is available, XShmQueryVersion
also returns ``major'' and ``minor'' which are the version
numbers of the extension implementation, and ``pixmaps''
which is True iff shared memory pixmaps are supported.
1m4.  USE OF SHARED MEMORY XIMAGES0m
The basic sequence of operations for shared memory XImages
is as follows:

\begin{enumerate}
\item Create the shared memory XImage structure
\item Create a shared memory segment to store the image
data
\item Inform the server about the shared memory segment
\item Use the shared memory XImage, much like a normal
\end{enumerate}

To create a shared memory XImage, use:

\begin{verbatim}
     XImage *XShmCreateImage (display, visual, depth, format, data,
			      shminfo, width, height)
	     Display *display;
	     Visual *visual;
	     unsigned int depth, width, height;
	     int format;
	     char *data;
	     XShmSegmentInfo *shminfo;
\end{verbatim}

Most of the arguments are the same as for XCreateImage; I
will not go through them here.	Note, however, that there
are no ``offset'', ``bitmap\_pad'', or ``bytes\_per\_line''
arguments.  These quantities will be defined by the server
itself, and your code needs to abide by them.  Unless you
have already allocated the shared memory segment (see
below), you should pass in NULL for the ``data'' pointer.
There is one additional argument: ``shminfo'', which is a
pointer to a structure of type XShmSegmentInfo.	 You must
allocate one of these structures such that it will have a
lifetime at least as long as that of the shared memory XIm-
age.  There is no need to initialize this structure before
the call to XShmCreateImage.

The return value, if all goes well, will be an XImage struc-
ture, which you can use for the subsequent steps.
The next step is to create the shared memory segment.  This
is best done after the creation of the XImage, since you
need to make use of the information in that XImage to know
how much memory to allocate.  To create the segment, you
need a call like:

\begin{verbatim}
     shminfo.shmid = shmget (IPC_PRIVATE,
	       image->bytes_per_line * image->height, IPC_CREAT|0777);
\end{verbatim}

(assuming that you have called your shared memory XImage
``image'').  You should, of course, follow the Rules and do
error checking on all of these system calls.  Also, be sure
to use the bytes\_per\_line field, not the width you used to
create the XImage as they may well be different.

Note that the shared memory ID returned by the system is
stored in the shminfo structure.  The server will need that
ID to attach itself to the segment.

Also note that, on many systems for security reasons, the X
server will only accept to attach to the shared memory seg-
ment if it's readable and writeable by ``other''. On systems
where the X server is able to determine the uid of the X
client over a local transport, the shared memory segment can
be readable and writeable only by the uid of the client.
Next, attach this shared memory segment to your process:

\begin{verbatim}
     shminfo.shmaddr = image->data = shmat (shminfo.shmid, 0, 0);
\end{verbatim}

The address returned by shmat should be stored in *both* the
XImage structure and the shminfo structure.

To finish filling in the shminfo structure, you need to
decide how you want the server to attach to the shared mem-
ory segment, and set the ``readOnly'' field as follows.
Normally, you would code:

\begin{verbatim}
     shminfo.readOnly = False;
\end{verbatim}

If you set it to True, the server will not be able to write
to this segment, and thus XShmGetImage calls will fail.
Finally, tell the server to attach to your shared memory
segment with:

\begin{verbatim}
     Status XShmAttach (display, shminfo);
\end{verbatim}

If all goes well, you will get a non-zero status back, and
your XImage is ready for use.
To write a shared memory XImage into an X drawable, use
XShmPutImage:

\begin{verbatim}
     Status XShmPutImage (display, d, gc, image, src_x, src_y,
			  dest_x, dest_y, width, height, send_event)
	     Display *display;
	     Drawable d;
	     GC gc;
	     XImage *image;
	     int src_x, src_y, dest_x, dest_y;
	     unsigned int width, height;
	     bool send_event;
\end{verbatim}

The interface is identical to that of XPutImage, so I will
spare my fingers and not repeat that documentation here.
There is one additional parameter, however, called
``send\_event''.	 If this parameter is passed as True, the
server will generate a ``completion'' event when the image
write is complete; thus your program can know when it is
safe to begin manipulating the shared memory segment again.
The completion event has type XShmCompletionEvent, which is
defined as the following:
\begin{verbatim}
     typedef struct {
	 int   type;		  /* of event */
	 unsigned long serial;	 /* # of last request processed */
	 Bool send_event;	  /* true if came from a SendEvent request */
	 Display *display;	  /* Display the event was read from */
	 Drawable drawable;	  /* drawable of request */
	 int major_code;     /* ShmReqCode */
	 int minor_code;     /* X_ShmPutImage */
	 ShmSeg shmseg;	     /* the ShmSeg used in the request */
	 unsigned long offset;	 /* the offset into ShmSeg used */
     } XShmCompletionEvent;
\end{verbatim}
The event type value that will be used can be determined at
run time with a line of the form:
\begin{verbatim}
     int CompletionType = XShmGetEventBase (display) + ShmCompletion;
\end{verbatim}
If you modify the shared memory segment before the arrival
of the completion event, the results you see on the screen
may be inconsistent.
To read image data into a shared memory XImage, use the fol-
lowing:
\begin{verbatim}
     Status XShmGetImage (display, d, image, x, y, plane_mask)
	  Display *display;
	  Drawable d;
	  XImage *image;
	  int x, y;
	  unsigned long plane_mask;
\end{verbatim}
Where ``display'' is the display of interest, ``d'' is the
source drawable, ``image'' is the destination XImage, ``x''
and ``y'' are the offsets within ``d'', and ``plane\_mask''
defines which planes are to be read.
To destroy a shared memory XImage, you should first instruct
the server to detach from it, then destroy the segment
itself, as follows:

\begin{verbatim}
     XShmDetach (display, shminfo);
     XDestroyImage (image);
     shmdt (shminfo.shmaddr);
     shmctl (shminfo.shmid, IPC_RMID, 0);
\end{verbatim}

\section{USE OF SHARED MEMORY PIXMAPS}
Unlike X images, for which any image format is usable, the
shared memory extension supports only a single format (i.e.
XYPixmap or ZPixmap) for the data stored in a shared memory
pixmap.	 This format is independent of the depth of the
image (for 1-bit pixmaps it doesn't really matter what this
format is) and independent of the screen.  Use XShmPixmap-
Format to get the format for the server:

\begin{verbatim}
     int XShmPixmapFormat (display)
	     Display *display;
\end{verbatim}

If your application can deal with the server pixmap data
format (including bits-per-pixel et al.), create a shared
memory segment and ``shminfo'' structure in exactly the same
way as is listed above for shared memory XImages.  While it
is, not strictly necessary to create an XImage first, doing
so incurs little overhead and will give you an appropriate
bytes\_per\_line value to use.
Once you have your shminfo structure filled in, simply call:

\begin{verbatim}
     Pixmap XShmCreatePixmap (display, d, data, shminfo, width,
			      height, depth);
	     Display *display;
	     Drawable d;
	     char *data;
	     XShmSegmentInfo *shminfo;
	     unsigned int width, height, depth;
\end{verbatim}

The arguments are all the same as for XCreatePixmap, with
two additions: ``data'' and ``shminfo''.  The second of the
two is the same old shminfo structure that has been used
before.	 The first is the pointer to the shared memory seg-
ment, and should be the same as the shminfo.shmaddr field.
I am not sure why this is a separate parameter.

If everything works, you will get back a pixmap, which you
can manipulate in all of the usual ways, with the added
bonus of being able to tweak its contents directly through
the shared memory segment.  Shared memory pixmaps are
destroyed in the usual manner with XFreePixmap, though you
should detach and destroy the shared memory segment itself
as shown above.

\begin{appendix}
\section{License}
\begin{center}
	      Copyright (C) 1991 X Consortium
\end{center}
Permission is hereby granted, free of charge, to any person
obtaining a copy of this software and associated documenta-
tion files (the ``Software''), to deal in the Software with-
out restriction, including without limitation the rights to
use, copy, modify, merge, publish, distribute, sublicense,
and/or sell copies of the Software, and to permit persons to
whom the Software is furnished to do so, subject to the fol-
lowing conditions:

The above copyright notice and this permission notice shall
be included in all copies or substantial portions of the
Software.
THE SOFTWARE IS PROVIDED ``AS IS'', WITHOUT WARRANTY OF ANY
KIND, EXPRESS OR IMPLIED, INCLUDING BUT NOT LIMITED TO THE
WARRANTIES OF MERCHANTABILITY, FITNESS FOR A PARTICULAR PUR-
POSE AND NONINFRINGEMENT.  IN NO EVENT SHALL THE X CONSOR-
TIUM BE LIABLE FOR ANY CLAIM, DAMAGES OR OTHER LIABILITY,
WHETHER IN AN ACTION OF CONTRACT, TORT OR OTHERWISE, ARISING
FROM, OUT OF OR IN CONNECTION WITH THE SOFTWARE OR THE USE
OR OTHER DEALINGS IN THE SOFTWARE.
Except as contained in this notice, the name of the X Con-
sortium shall not be used in advertising or otherwise to
promote the sale, use or other dealings in this Software
without prior written authorization from the X Consortium.
\end{appendix}
\end{document}
